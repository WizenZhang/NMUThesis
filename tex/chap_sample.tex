% 本LaTeX模板的使用示例
\chapter{示例}
%==============================
\section{参考文献引用}
参考文献类型:专著[M],会议论文集[C],报纸文章[N],期刊文章[J],学位论文[D],报告[R],标准[S],专利[P],论文集中的析出文献[A]。测试一下上标引用\upcite{Le2016Multiple},引用\cite{Le2016Multiple,Kaya2015,tf2017},还有其它引用\upcite{Li2017An,Le2016Multiple,tf2017}.
%--------------------------------
\subsection{数字标注}
\noindent
\begin{tabular}{l@{\quad$\Rightarrow$\quad}l}
	\verb|\cite{Li2017An}| & \cite{Li2017An}\\
	\verb|\citet{Li2017An}| & \citet{Li2017An}\\
	\verb|\citet[chap.~2]{Li2017An}| & \citet[chap.~2]{Li2017An}\\[0.5ex]
	\verb|\citep{Li2017An}| & \citep{Li2017An}\\
	\verb|\citep[chap.~2]{Li2017An}| & \citep[chap.~2]{Li2017An}\\
	\verb|\citep[see][]{Li2017An}| & \citep[see][]{Li2017An}\\
	\verb|\citep[see][chap.~2]{Li2017An}| & \citep[see][chap.~2]{Li2017An}\\[0.5ex]
	\verb|\citet*{Li2017An}| & \citet*{Li2017An}\\
	\verb|\citep*{Li2017An}| & \citep*{Li2017An}\\
\end{tabular}
\par\noindent
\begin{tabular}{l@{\quad$\Rightarrow$\quad}l}
	\verb|\citet{Li2017An,Kaya2015}| & \citet{Li2017An,Kaya2015}\\
	\verb|\citep{Li2017An,Kaya2015}| & \citep{Li2017An,Kaya2015}\\
	\verb|\cite{Li2017An,lamport1994}| & \cite{Li2017An,lamport1994}\\
	\verb|\upcite{Li2017An,lamport1994}| & \upcite{Li2017An,lamport1994}\\
	\verb|\citet{Li2017An,lamport1994}| & \citet{Li2017An,lamport1994}\\
	\verb|\citep{Li2017An,lamport1994}| & \citep{Li2017An,lamport1994}\\
	\verb|\cite{Li2017An,lamport1994,Kaya2015}| & \cite{Li2017An,lamport1994,Kaya2015}\\
\end{tabular}

%--------------------------------
\subsection{数字标注-上标形式}
\noindent
\begin{tabular}{l@{\quad$\Rightarrow$\quad}l}
	\verb|\upcite{Li2017An}| & \upcite{Li2017An}\\
	\verb|\upcite{Li2017An,lamport1994,Kaya2015}| & \upcite{Li2017An,lamport1994,Kaya2015}\\
\end{tabular}
\par\noindent
实现源码:\verb|\newcommand{\upcite}[1]{\textsuperscript{\cite{#1}}}|。


%--------------------------------
\subsection{著者-出版年制标}
\citestyle{authoryear}
\noindent
\begin{tabular}{l@{\quad$\Rightarrow$\quad}l}
	\verb|\cite{Li2017An}| & \cite{Li2017An}\\
	\verb|\citet{Li2017An}| & \citet{Li2017An}\\
	\verb|\citet[chap.~2]{Li2017An}| & \citet[chap.~2]{Li2017An}\\[0.5ex]
	\verb|\citep{Li2017An}| & \citep{Li2017An}\\
	\verb|\citep[chap.~2]{Li2017An}| & \citep[chap.~2]{Li2017An}\\
	\verb|\citep[see][]{Li2017An}| & \citep[see][]{Li2017An}\\
	\verb|\citep[see][chap.~2]{Li2017An}| & \citep[see][chap.~2]{Li2017An}\\[0.5ex]
	\verb|\citet*{Li2017An}| & \citet*{Li2017An}\\
	\verb|\citep*{Li2017An}| & \citep*{Li2017An}\\
\end{tabular}
\par\noindent
\begin{tabular}{l@{\quad$\Rightarrow$\quad}l}
	\verb|\citet{Li2017An,Kaya2015}| & \citet{Li2017An,Kaya2015}\\
	\verb|\citep{Li2017An,Kaya2015}| & \citep{Li2017An,Kaya2015}\\
	\verb|\cite{Li2017An,lamport1994}| & \cite{Li2017An,lamport1994}\\
	\verb|\citet{Li2017An,lamport1994}| & \citet{Li2017An,lamport1994}\\
	\verb|\citep{Li2017An,lamport1994}| & \citep{Li2017An,lamport1994}\\
\end{tabular}
\citestyle{numbers}

%--------------------------------
\subsection{其他形式的标注}
\noindent
\begin{tabular}{l@{\quad$\Rightarrow$\quad}l}
	\verb|\citealt{Kaya2015}| & \citealt{Kaya2015}\\
	\verb|\citealt*{Kaya2015}| & \citealt*{Kaya2015}\\
	\verb|\citealp{Kaya2015}| & \citealp{Kaya2015}\\
	\verb|\citealp*{Kaya2015}| & \citealp*{Kaya2015}\\
	\verb|\citealp{Kaya2015,Li2017An}| & \citealp{Kaya2015,Li2017An}\\
	\verb|\citealp[pg.~32]{Kaya2015}| & \citealp[pg.~32]{Kaya2015}\\
	\verb|\citenum{Kaya2015}| & \citenum{Kaya2015}\\
	\verb|\citetext{priv.\ comm.}| & \citetext{priv.\ comm.}\\
\end{tabular}

\noindent
\begin{tabular}{l@{\quad$\Rightarrow$\quad}l}
	\verb|\citeauthor{Kaya2015}| & \citeauthor{Kaya2015}\\
	\verb|\citeauthor*{Kaya2015}| & \citeauthor*{Kaya2015}\\
	\verb|\citeyear{Kaya2015}| & \citeyear{Kaya2015}\\
	\verb|\citeyearpar{Kaya2015}| & \citeyearpar{Kaya2015}\\
\end{tabular}

\section{浮动体\footnote{样例参考《浙江大学研究生硕士(博士)学位论文\LaTeX{}模板》}}
在实际撰写文稿的过程中,我们可能会碰到一些占据篇幅较大,但同时又不方便分页的内容。(比如图片和表格,通常属于这样的类型)此时,我们通常会希望将它们放在别的地方,避免页面空间不够而强行置入这些内容导致 overfull vbox 或者大片的空白。此外,因为被放在别的地方,所以,我们通常需要对这些内容做一个简单的描述,确保读者在看到这些大块的内容时,不至于无从下手去理解。同时,因为此类内容被放在别的地方,所以在文中引述它们时,我们无法用「下图」、「上表」之类的相对位置来引述他们。于是,我们需要对它们进行编号,方便在文中引用。

在 \LaTeX{} 中,默认有 figure 和 table 两种浮动体。(当然,你还可以自定义其他类型的浮动体)在这些环境中,可以用 $\backslash$caption\{\} 命令生成上述简短的描述。至于编号,也是用 $\backslash$caption\{\} 生成的。这类编号遵循了 TeX 对于编号处理的传统:它们会自动编号,不需要用户操心具体的编号数值。 至于「别的地方」是哪里,\LaTeX{} 为浮动体启用了所谓「位置描述符」的标记。基本来说,包含以下几种:

h - 表示 here。此类浮动体称为文中的浮动体(in-text floats)。

t - 表示 top。此类浮动体会尝试放在一页的顶部。

b - 表示 bottom。此类浮动体会尝试放在一页的底部。

p - 表示 float page,浮动页。此类浮动体会尝试单独成页。

\LaTeX{} 会将浮动体与文本流分离,而后按照位置描述符,根据相应的算法插入 \LaTeX{} 认为合适的位置。

\subsection{插图测试}
如图\ref{fig:first_image_tset}是对此模版的第一张插图测试。
\begin{figure}[htb]
	\centering
	\includegraphics[width = 5cm]{Chapter1.png}
	\caption{第一张插图测试}\label{fig:first_image_tset}
\end{figure}
以下是一段对这些插图来历的介绍,引用自知乎专栏All about TeXnique中夏晓昊的文章\href{http://zhuanlan.zhihu.com/LaTeX/19669122}{《The TeXbook导读:从那头(多图杀猫的)狮子说起》}。

在The TeXbook中,有着一系列的以狮子为主题的插图。这些插图的作者是Duane Bibby。也是从The TeXbook开始,不少TeX书也采取了以狮子为主的插图,作者也是Duane Bibby。另外,每年的TUG(TeX Users Group)年会都会有一张以狮子为主题的logo,这只狮子已经是社区的吉祥物了。

为什么选择狮子呢?Yannis Haralambous写道(原文法语,此为转译后的英文):Not for nothing is TeX represented by a lion. Donald Knuth has told us that lions are to him the guardians of libraries in the United States because there is a statue of a lion in front of the entrance of each large library there. Guardian of libraries, guardian of the Book—is that not indeed what TeX ultimately aspires to be? 或许吧。 (顺便说一句,TeX和MetaFont都用了狮子,TeX是公狮子,MetaFont是母狮子,多么和谐的一对啊。如果你还是忽略MetaFont的存在,那你还没有认识到它的重要性。)

作为插图,首要的一点就是贴切,然后是有趣。在TeX社区里面,have fun是一个很重要的词组,也有人说Happy TeXing。我知道有不少人不喜欢TeX,但是能有什么理由呢?如果你用不到它,那么浅尝辄止即可。如果你会用到很频繁,最好慢慢修炼做到精通。如果你只是偶尔用到,那么可以搬个模版什么的,甚至也可以找人帮你(不要指望别人会用足够的空闲时间来帮你,他没有这个义务,请支付报酬,最少也得请吃个饭吧)。下面的插图,是TeX TeXbook中的,我也希望这个假期,能有人有空来看看这本书。即使不能把所有的东西都看懂,那么也会对TeX的设计有了一定的了解,拿到扳手就好。

\subsection{表格测试}
在这里推荐制表采用功能强大的tabu宏包以取代其它制表宏包。具体tabu宏包的使用说明参见tabu宏包的说明文档。

以下节分别用来测试各种表格环境如,tabular,tabu,longtabu等,还有对caption格式的修改和测试。以下表格样式全部采用三线表。

\subsection{array宏包tabular表格环境测试}
如表\ref{tab:first_table_test}是对array宏包的tabular表格环境测试。

\begin{table}[htb]
	\centering
	\caption{tabular环境的编程语言优缺点对比表格}\label{tab:first_table_test}
	\begin{tabular}{lrr}
		\toprule
		\textbf{编程语言}     & \textbf{优势} & \textbf{劣势} \\
		\midrule
		Python     & 简单易学  & 速度较慢 \\
		C/C++     & 跨平台性能好    & 学习难度大 \\
		Java     & 使用范围最广 & 占用内存大 \\
		\bottomrule
	\end{tabular}%
\end{table}

\subsection{tabu宏包表格环境测试}
如表\ref{tab:tabu_test_1}是对tabu宏包的tabu表格环境测试。在这里表格命令与表\ref{tab:first_table_test}的命令相同,只是tabular环境改成了tabu环境。
\begin{table}[htb]
	\centering
	\caption{tabu环境的编程语言优缺点对比表格}\label{tab:tabu_test_1}
	\begin{tabu}{lrr}
		\toprule
		\textbf{编程语言}     & \textbf{优势} & \textbf{劣势} \\
		\midrule
		C$\#$     & 全面集成.Net库  & 跨平台能力太差 \\
		JavaScript     & 学习难度低    & 过于依赖浏览器 \\
		SQL    & 开发速度快,安全性好 & 运行速度换开发速度 \\
		\bottomrule
	\end{tabu}%
\end{table}

表\ref{tab:tabu_test_2}对tabu to表格的x列模式进行测试。在表格导言区中设置为X[1]X[2]X[2],表示这三列表格的列宽比值为1:2:2,总的表格宽度由tabu to环境设置,这里设置为0.6\textbackslash linewidth。相比于tabular环境,tabu环境的列宽设置方便许多。
\begin{table}[htb]
	\centering
	\caption{tabu环境的编程语言优缺点对比表格---X列模式}\label{tab:tabu_test_2}
	\begin{tabu} to 0.6\linewidth{X[1]X[2]X[2]}
		\toprule
		\textbf{编程语言}     & \textbf{优势} & \textbf{劣势} \\
		\midrule
		PHP     & 社区庞大活跃易上手  & 运行速度慢 \\
		Kotlin     & 和Java互操作性佳    & 继承了Java劣势 \\
		Swift     & 在iOS占比变大 & 版本更迭快差异大 \\
		\bottomrule
	\end{tabu}%
\end{table}

如表\ref{tab:tabu_test_3}是longtabu环境测试表格。longtabu环境不能用在table浮动体环境中。根据GB/T 7713.1-2006规定:如果某个表需要转页接排,在随后的各页上应重复表的编号。编号后跟标题(可省略)和“(续)”,置于表上方。续表应重复表头。

特别需要注意的是,longtabu是基于longtable宏包开发的,所以在nmu.cls文件中已经插入了longtable宏包。longtable环境的所有功能都可以在longtabu中使用,如\textbackslash endhead,\textbackslash endfirsthead,\textbackslash endfoot,\textbackslash endlastfoot,和\textbackslash caption等。具体用法请参见longtable和tabu宏包的相应文档。
\begin{longtabu}{cccccc}
	\caption{2018年6月全球编程语言TIOBE排行榜}\label{tab:tabu_test_3}\\
	\toprule
	Jun 2018   & Jun 2017 & Change & Programming Language & Ratings &Change\\
	\midrule%
	\endfirsthead
	\caption{2018年6月全球编程语言TIOBE排行榜(续)}\\
	\toprule
	Jun 2018   & Jun 2017 & Change & Programming Language & Ratings &Change \\
	\midrule%
	\endhead
	\bottomrule%
	\endfoot
	1	&	1	&		&	Java	&	15.368$\%$	&	+0.88$\%$	\\
	2	&	2	&		&	C	&	14.936$\%$	&	+8.09$\%$	\\
	3	&	3	&		&	C++	&	8.337$\%$	&	+2.61$\%$	\\
	4	&	4	&		&	Python	&	5.761$\%$	&	+1.43$\%$	\\
	5	&	5	&		&	C$\#$	&	4.314$\%$	&	+0.78$\%$	\\
	6	&	6	&		&	Visual Basic .NET	&	3.762$\%$	&	+0.65$\%$	\\
	7	&	8	&	$\uparrow$	&	PHP	&	2.881$\%$	&	+0.11$\%$	\\
	8	&	7	&	$\downarrow$	&	JavaScript	&	2.495$\%$	&	-0.53$\%$	\\
	9	&	-	&	$\uparrow$	&	SQL	&	2.339$\%$	&	+2,34$\%$	\\
	10	&	14	&	$\uparrow$	&	R	&	1.452$\%$	&	-0.70$\%$	\\
	11	&	11	&		&	Ruby	&	1.253$\%$	&	-0.97$\%$	\\
	12	&	18	&	$\uparrow$	&	Objective-C	&	1.181$\%$	&	-0.78$\%$	\\
	13	&	16	&	$\uparrow$	&	Visual Basic	&	1.154$\%$	&	-0.86$\%$	\\
	14	&	9	&	$\downarrow$	&	Perl	&	1.147$\%$	&	-1.16$\%$	\\
	15	&	12	&	$\downarrow$	&	Swift	&	1.145$\%$	&	-1.06$\%$	\\
	16	&	10	&	$\downarrow$	&	Assembly language	&	0.915$\%$	&	-1.34$\%$	\\
	17	&	17	&		&	MATLAB	&	0.894$\%$	&	-1.10$\%$	\\
	18	&	15	&	$\downarrow$	&	Go	&	0.879$\%$	&	-1.17$\%$	\\
	19	&	13	&	$\downarrow$	&	Delphi/Object Pascal	&	0.875$\%$	&	-1.28$\%$	\\
	20	&	20	&		&	PL/SQL	&	0.848$\%$	&	-0.72$\%$	\\
	21	&		&		&	SAS	&	1.102$\%$	&		\\
	22	&		&		&	Dart	&	0.799$\%$	&		\\
	23	&		&		&	COBOL	&	0.685$\%$	&		\\
	24	&		&		&	D	&	0.545$\%$	&		\\
	25	&		&		&	Lua	&	0.519$\%$	&		\\
	26	&		&		&	ABAP	&	0.463$\%$	&		\\
	27	&		&		&	Fortran	&	0.459$\%$	&		\\
	28	&		&		&	Transact-SQL	&	0.427$\%$	&		\\
	29	&		&		&	Scratch	&	0.398$\%$	&		\\
	30	&		&		&	Scala	&	0.377$\%$	&		\\
\end{longtabu}%

\subsection{子图}
这里子图的排版推荐使用subcaption宏包,不再推荐使用subfig宏包,更不推荐使用subfigure宏包。值得注意的是,在nmu.cls文件中已经写入了subcaption宏包,而且subcaption宏包与subfigure和subfig宏包是相互冲突的。因此,如果你还想使用subfig宏包而不想使用subcaption宏包,请自己到nmu.cls文件的相关位置更改,具体的使用及修改方法参见相应的宏包说明文档。不过在这里还是不推荐直接去更改nmu.cls文档,除非你对\LaTeX{} 的相关命令很清楚,知道自己在改什么,并且不会对其他格式产生影响。

具体的subcaption宏包使用方法我这里不仔细介绍,以下只是对subcaption进行一些简单的测试,主要是格式调整和交叉引用。

如图\ref{fig:subfig_test1}是有两张子图的模式,对子图进行交叉引用,如图\ref{subfig:1a}和图\ref{subfig:1b}。

\begin{figure}[htbp]
	\centering
	\begin{subfigure}[b]{.4\textwidth}
		\centering
		\includegraphics[width = \textwidth]{Chapter2.png}
		\caption{书籍排版与普通排版}\label{subfig:1a}
	\end{subfigure}
	\quad
	\begin{subfigure}[b]{.4\textwidth}
		\centering
		\includegraphics[width = \textwidth]{Chapter3.png}
		\caption{\TeX 的控制系列}\label{subfig:1b}
	\end{subfigure}
	\caption{子图模式测试1:2张图}\label{fig:subfig_test1}
\end{figure}

如图\ref{fig:subfig_test2}是有四张子图的模式,对子图进行交叉引用,如图\ref{subfig:2a}、图\ref{subfig:2b}、图\ref{subfig:2c}和图\ref{subfig:2d}。

\begin{figure}[htbp]
	\centering
	\begin{subfigure}[b]{.4\textwidth}
		\centering
		\includegraphics[width = \textwidth]{Chapter4.png}
		\caption{字体}\label{subfig:2a}
	\end{subfigure}
	\begin{subfigure}[b]{.4\textwidth}
		\centering
		\includegraphics[width = \textwidth]{Chapter5.png}
		\caption{编组}\label{subfig:2b}
	\end{subfigure}
	\begin{subfigure}[b]{.4\textwidth}
		\centering
		\includegraphics[width = \textwidth]{Chapter6.png}
		\caption{运行\TeX}\label{subfig:2c}
	\end{subfigure}
	\begin{subfigure}[b]{.4\textwidth}
		\centering
		\includegraphics[width = \textwidth]{Chapter7.png}
		\caption{\TeX 工作原理}\label{subfig:2d}
	\end{subfigure}
	\caption{子图模式测试2:4张图}\label{fig:subfig_test2}
\end{figure}

\section{算法环境}

模板中使用 \texttt{algorithm2e} 宏包实现算法环境。关于该宏包的具体用法请阅读宏包的官方文档。

\begin{algorithm}[!h]
	%\SetAlgoLined
	%\SetAlgoVlined
	\caption{A How to (plain).}
	\KwData{this text}
	\KwResult{how to write algorithm with \LaTeX2e{} }
	
	initialization\;
	\While{not at end of this document}{
		read current\;
		\eIf{understand}{
			go to next section\;
			current section becomes this one\;
		}{
			go back to the beginning of current section\;
		}
	}
\end{algorithm}

\RestyleAlgo{ruled}
\begin{algorithm}[!h]
	\caption{A How to (ruled).}
	\KwData{this text}
	\KwResult{how to write algorithm with \LaTeX2e{} }
	
	initialization\;
	\While{not at end of this document}{
		read current\;
		\eIf{understand}{
			go to next section\;
			current section becomes this one\;
		}{
			go back to the beginning of current section\;
		}
	}
\end{algorithm}


\RestyleAlgo{boxed}
\begin{algorithm}[!h]
	\caption{A How to (boxed).}
	\KwData{this text}
	\KwResult{how to write algorithm with \LaTeX2e{} }
	
	initialization\;
	\While{not at end of this document}{
		read current\;
		\eIf{understand}{
			go to next section\;
			current section becomes this one\;
		}{
			go back to the beginning of current section\;
		}
	}
\end{algorithm}

\RestyleAlgo{boxruled}
\begin{algorithm}[!h]
	\caption{PARTITION$(A,p,r)$ (boxruled)}%算法标题
	\KwData{this text}
	\KwResult{how to write algorithm with \LaTeX2e{} }
	\begin{algorithmic}[1]%一行一个标行号
		\STATE $i=p$
		\FOR{$j=p$ to $r$}
		\IF{$A[j]<=0$}
		\STATE $swap(A[i],A[j])$
		\STATE $i=i+1$
		\ENDIF
		\ENDFOR
	\end{algorithmic}
\end{algorithm}

\section{代码环境}
\subsection{全局设置}
listings 是专用于代码排版的\LaTeX{}宏包,可对关键词、注释和字符串等使用不同的字体和颜色或颜色,也可以为代码添加边框、背景等风格。很多时候需要对文档中的代码进行解释,只有带有行号的代码才可以让解释更清晰,因为你只需要说第 x行代码有什么作用即可。如果没有行号,那对读者而言就太残忍了,他们不得不从你的文字叙述中得知行号信息,然后去一行一行的查到相应代码行。listings 宏包通过参数 numbers 来设定行号,该参数的值有两个,分别是 left 与right,表示行号显示在代码的左侧还是右侧。下面给出一份用于排版C语言程序代码样例如图\ref{fig:code}所示:
\begin{figure}[htb!]
	\centering
	\begin{lstlisting}[language={[ANSI]C}] 
int main(int argc, char ** argv) 
{ 
    /*格式化并输出结果到标准输出*/
    printf("`我爱TeXing`! \n"); 
    return 0; 
} 
	\end{lstlisting} 
	\caption{C语言程序代码样例}
	\label{fig:code}
\end{figure}

\subsection{显示中文}
listings 宏包默认是不支持包含中文字串的代码显示的,但是可以使用 “逃逸” 字串来显示中文。

在 \verb|\lstset| 命令中设置逃逸字串的开始符号与终止符号,推荐使用的符号是左引号,即 “ `”。

\section{流程图}

\subsection{流程图概述}

流程图是流经一个系统的信息流、观点流或部件流的图形代表。在企业中,流程图主要用来说明某一过程。这种过程既可以是生产线上的工艺流程,也可以是完成一项任务必需的管理过程。流程图是揭示和掌握封闭系统运动状况的有效方式。作为诊断工具,它能够辅助决策制定,让管理者清楚地知道,问题可能出在什么地方,从而确定出可供选择的行动方案。

流程图是表达算法思想最为有效的图形工具。作为计算机专业的学生,我们经常需要在文档中使用流程图来描述算法。在 LaTeX 中使用流程图可以通过 TikZ 或 flowchart 宏包来实现,但从本质上来说 flowchart 宏包也是使用 TikZ 宏包来实现的。flowchart 定义的形状数量比较少,可能满足不了绘制复杂流程图的需要,直接使用TikZ强大的绘图功能来实现流程图的绘制如图\ref{fig:chart},。

\begin{figure}[htb!]
	\centering
	\begin{tikzpicture}[node distance=1.8cm]
	%定义流程图具体形状
	\node (start) [startstop] {Start};
	\node (in1) [io, below of=start] {Input};
	\node (pro1) [process, below of=in1] {Process 1};
	\node (dec1) [decision, below of=pro1, yshift=-0.5cm] {Decision 1};
	\node (dec2) [decision, below of=dec1, yshift=-1.5cm] {Decision 2};
	\node (pro2) [process, right of=dec1, xshift=3cm] {Process 2};
	\node (out1) [io, below of=dec2, yshift=-0.5cm] {Output};
	\node (stop) [startstop, below of=out1] {Stop};
	\coordinate (pointN) at (-3cm, -9.2cm);
	%连接具体形状
	\draw [arrow](start) -- (in1);
	\draw [arrow](in1) -- (pro1);
	\draw [arrow](pro1) -- (dec1);
	\draw [arrow](dec1) -- (dec2);
	\draw [arrow](dec1) -- (pro2);
	\draw [arrow](dec1) -- node[anchor=east] {Y} (dec2);
	\draw [arrow](dec1) -- node[anchor=south] {N} (pro2);
	\draw [arrow](pro2) |- (pro1);
	\draw [arrow](dec2) -- node[left] {Y} (out1);
	\draw (dec2) -- node[above] {N} (pointN);
	\draw [arrow](pointN) |- (pro1);
	\draw [arrow](out1) -- (stop);
	\end{tikzpicture}
	\caption{直接使用TikZ宏包绘制的流程图}
	\label{fig:chart}
\end{figure}

\subsection{流程图的形式}
流程图常用的形式有两种:
 
(1)上下流程图 

上下流程图是最常见的一种流程图,它仅表示上一步与下一步的顺序关系。
 
(2)矩阵流程图

矩阵流程图不仅表示下下关系,还可以看出某一过程的其他关系。最后给一个流程图的例子如图\ref{fig:flowsheet}。流程图的原图来自于Itti的显著性论文。里面基本包含了常用流程图画法中的所有要点。

\subsection{\LaTeX{}中插入VISIO图}

1. VISIO原图另存为PDF格式

2. 用Acrobat打开,文档 -- 剪裁页面 -- 删除白边距

3. 将pdf格式文件复制到figures文件夹(盲审封皮:\verb|\includegraphics\{nmu.pdf\}|)

4. 或另存为eps格式

5. 将eps格式文件复制到figures文件夹

\begin{figure}[htb!]
	\centering
	%定义形状样式
	\tikzstyle{startstop} = [rectangle, rounded corners, minimum width = 3cm, minimum height = 0.7cm, text centered, draw = black]
	\tikzstyle{startstop2} = [rectangle, rounded corners, minimum width = 13cm, minimum height = 0.7cm, text centered, draw = black]
	\tikzstyle{io} = [trapezium, trapezium left angle = 30, trapezium right angle = 150, minimum width = 3cm, text centered, draw = black, fill = white]
	\tikzstyle{io2} = [trapezium, trapezium left angle = 30, trapezium right angle = 150, minimum width = 2.5cm, draw = black, fill = white]
	\tikzstyle{io3} = [trapezium, trapezium left angle = 30, trapezium right angle = 150, minimum width = 2cm, draw = black, fill = white]
	\tikzstyle{process} = [rectangle, minimum width = 3cm, minimum height = 1cm, text centered, draw = black]
	\tikzstyle{decision} = [diamond, minimum width = 3cm, minimum height = 1cm, text centered, draw = black]
	\tikzstyle{arrow} = [thick, -, >= stealth]
	\tikzstyle{arrow2} = [thick, ->, >= stealth]
	
	\begin{tikzpicture}[node distance = 1.5cm]
	% 定义流程图具体形状
	\coordinate[label = left:{\small 输入图像}](A) at(-1.5, 0);
	\node(in1) [io] {};
	\node(pro1) [startstop, below of = in1] {\small 线性滤波};
	
	\node(in2 - 2)[io3, below of = pro1, yshift = -0.6cm]{};
	\node(in3 - 2)[io3, left of = in2 - 2, xshift = -2.5cm]{};
	\node(in4 - 2)[io3, right of = in2 - 2, xshift = 2.5cm]{};
	
	\node(in2 - 1)[io2, below of = pro1, yshift = -0.3cm]{};
	\node(in3 - 1)[io2, left of = in2 - 1, xshift = -2.5cm]{};
	\node(in4 - 1)[io2, right of = in2 - 1, xshift = 2.5cm]{};
	
	\node(in2) [io, below of = pro1] {\small 颜色};
	\node(in3)[io, left of = in2, xshift = -2.5cm]{ \small 亮度 };
	\node(in4)[io, right of = in2, xshift = 2.5cm]{ \small 方向 };
	
	\node(in5)[startstop2, below of = in2 - 2]{ \small Center - Surround差异计算及归一化 };
	
	\node(in6 - 2)[io3, below of = in5, yshift = -0.6cm]{};
	\node(in7 - 2)[io3, left of = in6 - 2, xshift = -2.5cm]{};
	\node(in8 - 2)[io3, right of = in6 - 2, xshift = 2.5cm]{};
	
	\node(in6 - 1)[io2, below of = in5, yshift = -0.3cm]{};
	\node(in7 - 1)[io2, left of = in6 - 1, xshift = -2.5cm]{};
	\node(in8 - 1)[io2, right of = in6 - 1, xshift = 2.5cm]{};
	
	\node(in6) [io, below of = in5] {};
	\node(in7)[io, left of = in6, xshift = -2.5cm]{};
	\node(in8)[io, right of = in6, xshift = 2.5cm]{};
	
	\coordinate[label = left:{\small 特征图}](B) at(-1, -6.2);
	\coordinate[label = left:{\small (12张)}](C) at(-1.5, -7.5);
	\coordinate[label = left:{\small (6张)}](D) at(2.7, -7.5);
	\coordinate[label = left:{\small (24张)}](E) at(6.7, -7.5);
	
	\node(in9)[startstop2, below of = in6 - 2]{ \small 跨尺度合并及归一化 };
	
	\node(in10) [io, below of = in9] {};
	\node(in11)[io, left of = in10, xshift = -2.5cm]{};
	\node(in12)[io, right of = in10, xshift = 2.5cm]{};
	
	\coordinate[label = left:{\small 醒目图}](F) at(-1, -9.5);
	\node(in13) [startstop, below of = in10] {\small 线性组合};
	\node(in14) [io, below of = in13] {};
	\coordinate[label = left:{\small 显著图}](G) at(-1, -13);
	
	\node(in15) [startstop, below of = in14] {\small 赢者取全};
	\coordinate[label = left:{\small 显著位置}]() at(1, -16.1);
	\coordinate[label = left:{\small 反馈抑制}]() at(4.5, -14.7);
	
	%连线
	\draw[arrow](pro1) -- (in1);
	\draw[arrow](pro1) -- (in2);
	\draw[arrow](pro1) -- (in3);
	\draw[arrow](pro1) -- (in4);
	\draw[arrow](0, -4.75) -- (in2 - 2);
	\draw[arrow](-4, -4.75) -- (in3 - 2);
	\draw[arrow](4, -4.75) -- (in4 - 2);
	\draw[arrow](0, -5.45) -- (in6);
	\draw[arrow](-4, -5.45) -- (in7);
	\draw[arrow](4, -5.45) -- (in8);
	\draw[arrow](0, -8.35) -- (in6 - 2);
	\draw[arrow](-4, -8.35) -- (in7 - 2);
	\draw[arrow](4, -8.35) -- (in8 - 2);
	\draw[arrow](0, -9.05) -- (in10);
	\draw[arrow](-4, -9.05) -- (in11);
	\draw[arrow](4, -9.05) -- (in12);
	\draw[arrow](in13) -- (in10);
	\draw[arrow](in13) -- (in11);
	\draw[arrow](in13) -- (in12);
	\draw[arrow](in13) -- (in14);
	\draw[arrow](in14) -- (in15);
	\draw[arrow](in15) -- (0, -15.8);
	\draw[arrow](0, -15.4) -- (2.5, -15.4);
	\draw[arrow](2.5, -14) -- (2.5, -15.4);
	\draw[arrow2](2.5, -14) -- (0, -14);
	\end{tikzpicture}
	\caption{图形图像处理模型流程图}
	\label{fig:flowsheet}
\end{figure}

\section{数学环境}

\subsection{数学符号}

模板定义了一些正体(upright)的数学符号:
\begin{center}
	\begin{tabular}{rl}
		\toprule
		符号                 & 命令 \\
		\midrule
		常数$\eu$     & \verb|\eu| \\
		复数单位$\iu$ & \verb|\iu| \\
		微分符号$\diff$ & \verb|\diff| \\
		$\argmax$         & \verb|\argmax| \\
		$\argmin$         & \verb|\argmin| \\
		\bottomrule
	\end{tabular}
\end{center}

更多的例子:
\begin{equation}
\eu^{\iu\pi} + 1 = 0
\end{equation}
\begin{equation}
\frac{\diff^2u}{\diff t^2} = \int f(x) \diff x
\end{equation}
\begin{equation}
\argmin_x f(x)
\end{equation}

\subsection{定理、引理和证明}

\begin{definition}
	A sigmoid function is a mathematical function having a characteristic "S"-shaped curve or sigmoid curve. Often, sigmoid function refers to the special case of the logistic function shown in the first figure and defined by the formula:
	\begin{equation}
	sigmoid(x) = \frac{1}{1 + e^{-x}}
	\end{equation}
	
	\begin{figure}[htb]
		\centering
		\begin{tikzpicture}
		\draw[->](-5.2,0)--(5.2,0)node[left,below]{$x$};
		\draw[->](0,-0.5)--(0,2)node[right]{$y$};
		\draw[dashed](-5,1)--(5,1);
		\foreach \x in {-4,-3,-2,-1,0,1,2,3,4}{\draw(\x,0)--(\x,0.05)node[below,outer sep=2pt,font=\small]at(\x,0){\x};}
		\foreach \y in {0.5,1}{\draw(0,\y)--(0.05,\y)node[left,outer sep=2pt,font=\small]at(0,\y){\y};}
		\draw[color=red ,domain=-4:4]plot(\x,{1/(1+(e^(-1*(\x))))});
		\end{tikzpicture}
	\end{figure}

	Special cases of the sigmoid function include the Gompertz curve (used in modeling systems that saturate at large values of x) and the ogee curve (used in the spillway of some dams). Sigmoid functions have domain of all real numbers, with return value monotonically increasing most often from 0 to 1 or alternatively from −1 to 1, depending on convention.
\end{definition}



\begin{example}
	Simple examples of functions on $\mathbf{R}^d$ that are integrable
	(or non-integrable) are given by
	\begin{equation}
	f_a(x) =
	\begin{cases}
	|x|^{-a} & \text{if } |x| \leq 1,\\
	0 & \text{if } x > 1.
	\end{cases}
	\end{equation}
	\begin{equation}
	F_a(x) = \frac{1}{1 + |x|^a}, \qquad \text{all } x \in \mathbf{R}^d.
	\end{equation}
	Then $f_a$ is integrable exactly when $a < d$, while $F_a$ is integrable
	exactly when $a > d$.
\end{example}

\begin{lemma}[Fatou]
	Suppose $\{f_n\}$ is a sequence of measurable functions with $f_n \geq 0$.
	If $\lim_{n \to \infty} f_n(x) = f(x)$ for a.e. $x$, then
	\begin{equation}
	\int f \leq \liminf_{n \to \infty} \int f_n.
	\end{equation}
\end{lemma}

\begin{remark}
	We do not exclude the cases $\int f = \infty$,
	or $\liminf_{n \to \infty} f_n = \infty$.
\end{remark}

\begin{corollary}
	Suppose $f$ is a non-negative measurable function, and $\{f_n\}$ a sequence
	of non-negative measurable functions with
	$f_n(x) \leq f(x)$ and $f_n(x) \to f(x)$ for almost every $x$. Then
	\begin{equation}
	\lim_{n \to \infty} \int f_n = \int f.
	\end{equation}
\end{corollary}

\begin{proposition}
	Suppose $f$ is integrable on $\mathbf{R}^d$. Then for every $\epsilon > 0$:
	\begin{enumerate}
		\renewcommand{\theenumi}{\roman{enumi}}
		\item There exists a set of finite measure $B$ (a ball, for example) such that
		\begin{equation}
		\int_{B^c} |f| < \epsilon.
		\end{equation}
		\item There is a $\delta > 0$ such that
		\begin{equation}
		\int_E |f| < \epsilon \qquad \text{whenever } m(E) < \delta.
		\end{equation}
	\end{enumerate}
\end{proposition}

\begin{theorem}
	Suppose $\{f_n\}$ is a sequence of measurable functions such that
	$f_n(x) \to f(x)$ a.e. $x$, as $n$ tends to infinity.
	If $|f_n(x)| \leq g(x)$, where $g$ is integrable, then
	\begin{equation}
	\int |f_n - f| \to 0 \qquad \text{as } n \to \infty,
	\end{equation}
	and consequently
	\begin{equation}
	\int f_n \to \int f \qquad \text{as } n \to \infty.
	\end{equation}
\end{theorem}

\begin{proof}
	Trivial.
\end{proof}


\subsection{自定义}

\newtheorem*{axiomofchoice}{Axiom of choice}
\begin{axiomofchoice}
	Suppose $E$ is a set and ${E_\alpha}$ is a collection of
	non-empty subsets of $E$. Then there is a function $\alpha
	\mapsto x_\alpha$ (a ``choice function'') such that
	\begin{equation}
	x_\alpha \in E_\alpha,\qquad \text{for all }\alpha.
	\end{equation}
\end{axiomofchoice}

\newtheorem{observation}{Observation}[chapter]
\begin{observation}
	Suppose a partially ordered set $P$ has the property
	that every chain has an upper bound in $P$. Then the
	set $P$ contains at least one maximal element.
\end{observation}
\begin{proof}[A concise proof]
	Obvious.
\end{proof}

\newtheorem{observationvar2}[observation]{Observationvar2}
\begin{observationvar2}
	Suppose a partially ordered set $P$ has the property
	that every chain has an upper bound in $P$. Then the
	set $P$ contains at least one maximal element.
\end{observationvar2}
\begin{proof}[A concise proof]
	Obvious.
\end{proof}
