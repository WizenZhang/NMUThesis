%% %%=================================================================
%% %% <UTF-8>
%% %% 北方民族大学学术论文模板使用样例
%% %% 请将以下文件与此LaTeX文件放在同一目录中.
%% %%-----------
%% %% nmu.cls              : LaTeX宏模板文件
%% %% GBT7714-2005.bst      : 国标参考文献BibTeX样式文件2005(https://github.com/Haixing-Hu/GBT7714-2005-BibTeX-Style)
%% %% GBT7714-2015.bst      : 国标参考文献BibTeX样式文件2015(https://github.com/zepinglee/gbt7714-bibtex-style)
%% %% nmu_logo.png          : 论文封皮北方民族大学校徽
%% %% tex/*.tex             : 本模板样例中的独立章节
%% %%-----------
%% %% 请统一使用UTF-8编码.
%% %%=================================================================

%=================================================================
\documentclass[master,onside,ultimate]{nmu}
%=================================================================
% nmu基于ctexbook模板
% 论文样式参考自院教字〔2003〕169号《北方民族大学研究生学位论文格式和要求》
% 本模板改写自《北京航空航天大学学术论文LaTeX模板》
%======================
% 模板选项:
%======================
% I.论文类型(thesis)
%--------------------
% a.学术硕士论文(master)<缺省值>
% b.专业硕士论文(professional)
% c.博士论文(doctor)
%--------------------
% II.打印设置(printtype)
%--------------------
% a.单面打印(onside)<缺省值>
% b.双面答应(twoside)
%--------------------
% III.论文版本设置(version)
%--------------------
% a.盲审版(blind)<缺省值>
% b.最终版(ultimate)
%--------------------
%=================================================================

%=================================================================
% 开启/关闭引用编号颜色:参考文献,公式,图,表,算法 等……
\refcolor{on}   % 开启: on<默认>; 关闭: off;
% 摘要和正文从右侧开始
\beginright{on} % 开启: on<默认>; 关闭: off;
% 空白页留字
\emptypagewords{[ -- This page is a preset empty page -- ]}

%=================================================================
% nmu模板已内嵌以下LaTeX工具包:
%--------------------
% ifthen, etoolbox, titletoc, remreset, remreset,
% geometry, fancyhdr, setspace, caption,
% float, graphicx, subfigure, epstopdf,
% booktabs, longtable, multirow, 
% array, enumitem
% algorithm2e, amsmath, amsthm, listings
% pifont, color, soul
%--------------------
% 请在此处添加额外工具包>> 


%=================================================================
% nmu模板已内嵌以下LaTeX宏:
%--------------------
% \highlight{text} % 黄色高亮
%--------------------
% 请在此处添加自定义宏>> 


%%=================================================================
% 论文标题
\title{北方民族大学研究生学术论文编写规则LATEX模板}
% 根据ulem的文档第五页说明,通过宏或者命令传递到uline 命令中的英文句子是被盒子装
% 起来的,如果不特殊处理,则句子无法断行,但是\\,\ ,和 \- 还是可以起作用,因此
% 须在\englishtitle命令中的句子的适当位置添加"\",使句子可以断行。中文则不存在此
% 问题。 

\englishtitle{LATEX Template For The Academic Dissertaion Of North Minzu University}
\author{学生姓名}
% 分类号
\classification{TP227}
% 单位代码
\serialnumber{11407}
% 密级
\secretlevel{公开}
% 学号
\studentnumber{20160000}
% 指导教师
\supervisor{导师姓名 教授}
% 申请的学位门类
\applyclass{工学硕士}
% 专业名称
\major{计算机系统结构}
% 研究方向
\research{嵌入式系统与物联网技术}
% 所在学院
\institute{计算机科学与工程学院}
% 提交日期
\submitdate{2019年3月}
% 毕业届(页眉)
\session{2019}
%%=================================================================
% 摘要-{中文}{英文}
\Abstract{%
  摘要是学术论文内容的简短陈述,应体现论文工作的核心思想。论文摘要应力求语言精炼准确。博士学术论文的中文摘要一般约800~1200字;硕士学位论文的中文摘要一般约500字。摘要内容应涉及本项科研工作的目的和意义、研究思想和方法、研究成果和结论。博士学位论文必须突出论文的创造性成果,硕士学位论文必须突出论文的新见解。
  
  关键字是为用户查找文献,从文中选取出来揭示全文主体内容的一组词语或术语,应尽量采用词表中的规范词(参考相应的技术术语标准)。关键词一般3~5个,按词条的外延层次排列(外延大的排在前面)。关键词之间用逗号分开,最后一个关键词后不打标点符号。
  
  为了国际交流的需要,论文必须有英文摘要。英文摘要的内容及关键词应与中文摘要及关键词一致,要符合英语语法,语句通顺,文字流畅。英文和汉语拼音一律为Times New Roman体,字号与中文摘要相同。
  }{%
  What were you doing 500 years ago? Oh, that’s right nothing, because you didn’t exist yet. In fact, several generations of your family had yet to leave their mark on the world, but one very special shark may already have been swimming in the chilly North Atlantic at that time, and the incredible animal is somehow still alive today.
  
  Scientists studying Greenland sharks observed the particularly old specimen just recently, and after studying it they’ve determined that the creature is approximately 272 to 512 years old. That’s an absolutely insane figure, and if its age lands towards the higher end, it makes the animal the oldest observed living vertebrate on the entire planet.
  

  
  The shark, which is a female, measures an impressive 18 feet long. That’s pretty large, but it might not sound particularly large for an ocean-dwelling creature that lives hundreds of years. That is, until you consider that the Greenland shark only grows around one centimeter per year. With that in mind, 18 feet is actually downright massive.
  
  As for how this particular shark species manages to live so incredibly long, scientists attribute a lot of its longevity to its sluggish metabolism, as well as its environment. The frigid waters where the sharks thrive is thought to increase overall lifespan in a variety of ways. Past research has shown that cold environments can help slow aging, and these centuries-old sharks are most certainly benefiting from their chilly surroundings.
  
  --- Online news {\it Scientists find incredible shark that may be over 500 years old and still kicking\/}, 12.16.2017. (http://bgr.com/2017/12/14/oldest-shark-greenland-512-years-old/).
}
% 关键字-{中文}{英文}
\Keyword{%
    北民大,学术论文,博士,硕士,学硕,专硕,中文,\LaTeX{}模板}{%
    NMU,Master, Doctor, Template
}

% 图标目录

\Listfigtab{off} % 启用: on<默认>; 关闭: off;



\begin{document}

%%=================================================================
% 标题级别
%--------------------
% \chapter{第一章}
% \section{1.1 小节}
% \subsection{1.1.1 条}
% \subsubsection{1.1.1.1}
% \paragraph{1.1.1.1.1}
% \subparagraph{1.1.1.1.1.1}
%--------------------
%%=================================================================
% 绪论
% [绪论]
% 此次为本LaTeX模板的简介
\chapter{绪论}
大家好,这是北方民族大学学术论文\LaTeX{}模板(\CTeX{}-Based)---\NMUThesis{}。

\NMUThesis{}为北民大研究生学术论文模板,适用于理工类学术硕士和专业硕士。本\LaTeX{}模板参考自院教字〔2003〕169号《北方民族大学研究生学位论文格式和要求》(以下简称《格式》),具体要求请参见《格式》,最终成文格式需参考学院要求及打印方意见。本模板中大量内容和说明直接摘抄自《格式》,基本覆盖了论文内容和格式方面的要求。

文献著录BibTeX样式采用Haixing Hu开源的2005版参考文献著录BibTeX样式\href{https://github.com/Haixing-Hu/GBT7714-2005-BibTeX-Style}{GBT7714-2005}及Zeping Lee开源的2015版参考文献著录BibTeX样式\href{https://github.com/zepinglee/gbt7714-bibtex-style}{GBT7714-2015},在此感谢两位的开源分享。请自行选用:\\
\verb|\bibliographystyle\{GBT7714-2005\}|或\\
\verb|\bibliographystyle\{GBT7714-2015\}|。

本模板改写自《北京航空航天大学学术论文LaTeX模板》,已上传至GitHub\footnote{\href{https://github.com/WizenZhang/NMUThesis}{https://github.com/WizenZhang/NMUThesis}}。

意见及问题反馈请联系:\\
\indent E-mail:wizen\_zhang@163.com\\
\indent GitHub:\href{https://github.com/WizenZhang/NMUThesis/issues}{https://github.com/WizenZhang/NMUThesis/issues}

%%============================
\section{概述}
硕士研究生学位论文是学位申请人为申请硕士学位而撰写的学术论文,它集中表明了作者在研究工作中获得的新成果,是评判学位申请人学术水平的重要依据和获得学位的必要条件之一,也是科研领域中的重要文献资料和社会的宝贵财富。为提高我校硕士学位论文的质量,规范学位论文格式,特作如下规定。

%%============================
\section{基本要求}

\begin{enumerate}[label=\arabic*)]
	\item 硕士学位论文应能表明作者确已在本门学科上掌握了坚实的基础理论和系统的专门知识,并对所研究课题有新的见解,有从事科学研究工作或独立担负专门技术工作的能力。
	
	\item 除外语专业外,学位论文一般用中文撰写,硕士学位论文正文应不少于2万字。学位论文内容应立论正确、推理严谨、文字简练、层次分明、说理透彻、数据真实可靠。
	
	\item 量和单位及其符号均应符合国家标准的规定,国家标准中未规定的,应执行国际标准或行业标准;不同的量必须用不同的符号表示,不得一符多义,含义相同的量则必须用同一符号表示。学位论文应用最新颁布的汉语简化文字,符合《出版物汉字使用管理规定》;专业术语应统一使用全国自然科学名词审定委员会公布的各学科名词,或本学科权威和期刊通用的专业术语,且前后应一致;标点符号的使用应符合国家标准《标点符号用法》的规定;数字的使用应符合国家标准《出版物上数字用法的规定》。
	
	\item 图要精选,切忌与文字或表内容重复,图中文字、数据和符号应准确无误且与文字叙述一致,图应有图名,图名应简洁明确且与图中内容相符。表应用表序和表名,表名应简洁并与内容相符。图、表和公式应分别顺序编号。
\end{enumerate}

论文内容包括:选题的背景、依据及意义;文献及相关研究综述、研究及设计方案、实验方法、装置和实验结果;理论的证明、分析和结论;重要的计算、数据、图表、曲线及相关分析;必要的附录、相关的参考文献目录等,如表\ref{tab:papercomponents}。

\centerline{-----------$\downarrow$-----------Space Check-----------$\downarrow$-----------}
\begin{table}[h]
  \caption{学术论文组成}
  \label{tab:papercomponents}
  \centering
  \begin{tabular}{cp{16\ccwd}p{4cm}}
    \toprule
    {\bfseries 装订顺序} & \multicolumn{1}{c} {\bfseries 内容} & \multicolumn{1}{c} {\bfseries 说明}  \\
    \midrule
    1 & 封面& \\
    3 & 独创性声明和使用授权书 & \\
    4 & 中文摘要        & \\
    5 & 英文摘要        & \\
    6 & 目录            & \\
    7 & 正文            & \\
    8 & 结论/结语	        & \\
    9 & 参考文献        & \\
    10& 附录            & 非必要 \\
    12& 致谢            & 盲审论文无此项 \\
    13& 个人简介        & 盲审论文无此项 \\
    \bottomrule
  \end{tabular}
\end{table}
\centerline{-----------$\uparrow$-----------Space Check-----------$\uparrow$-----------}
%%============================

\section{版式及其它要求}

%%============================

%%----------------------
\subsection{开本及版心}
{\bfseries 论文开本大小}:210mm×297mm(标准A4纸)。

{\bfseries 论文版心}:左边距:30mm,右边距:25mm,上边距:30mm,下边距:25mm,页眉边距:23mm,页脚边距:20mm。
%%----------------------
\subsection{页眉及页脚}

\begin{enumerate}[label=\arabic*)]
	\item 从正文开始各页均加有页眉、页脚,文字均采用小五号宋体。
	
	\item 页眉左侧为“北方民族大学$\times$ $\times$ $\times$ 届硕士学位论文”,右侧为一级标题名称;页眉下横线为上粗下细文武线(3磅)。
	
	\item 页码格式为“-1-”,单面打印时,插入的页码排在页脚居中的位置;双面打印时,插入的页码分别排在页脚左右侧。
	
	\item 从内封面到目录,均用英文页码,如“I、II、III”,从引言到论文末页,页码用阿拉伯数字,如“-1-、-2-、-3-”。
\end{enumerate}

%%----------------------
\subsection{封面}

\begin{enumerate}[label=\arabic*)]
	\item 论文内外封面内容一样,外封皮用草绿色暗纹纸。
	
	\item 论文题目中英文对照,均可分两行排列;中文用黑体二号字,英文用Times New Roman三号字。
	
	\item 分类号按《中国图书资料分类法》要求查询填写。
	
	\item 密级:涉密论文,学院学位评定分委员会根据国家规定的密级范围和法定程序审查确定,并注明相应的保密年限;不需保密的应填写“公开”。
	
	\item 论文完成日期统一用阿拉伯数字填写。
\end{enumerate}

%%----------------------

%%----------------------
\subsection{独创性声明和使用授权书}

独创性声明和关于论文使用授权的说明附于内封面后,需由研究生和指导教师本人签字。

%%----------------------
%%============================

\section{论文各组成部分要求}

%%============================
%%----------------------
\subsection{摘要及关键词}

\begin{enumerate}[label=\arabic*)]
	\item 摘要即摘录论文要点,是论文要点不加注释和评论的一篇完整的陈述性短文,具有很强的自含性和独立性,能独立使用和被引用。
	
	\item 摘要应含有学位论文全文的主要信息,一般包括研究目的、研究方法、所取得的结果和结论。论文摘要应突出新见解或创新性。
	
	\item 摘要的详简度视论文的内容、性质而定,硕士学位论文摘要一般为500-600字,但不能超过1000字。
	
	\item 摘要中一般不用图、表、化学结构式、计算机程序,不用非公知公用的符号、术语和非法定的计量单位。
	
	\item “摘要” 居中用三号黑体字,3倍行间距;“关键词” 另起一行置于摘要下方,左对齐,用四号宋体加粗;摘要和关键词的内容用小四号宋体,行间距为1.5倍行距。
	
	\item 摘要一般为3至5个,中间以“,”分隔,涉及的内容、领域从大到小排列,便于文献编目与查询。
	
	\item 应有与中文摘要和关键词相对应的英文摘要和关键词。英文摘要用词要准确使用本学科通用词汇;摘要中主语(作者)常常省略,因而一般使用被动语态;应使用正确的时态并要注意主谓语的一致,必要的冠词不能省略。
	
	\item 英文摘要和关键词字号与中文一样,用Times New Roman字体;涉及到的姓名、书名等用斜体。
\end{enumerate}

%%----------------------
\subsection{目录}

\begin{enumerate}[label=\arabic*)]
	\item 目录依论文内的章节标题次序排列,标题应该简明扼要。
	
	\item 目录中仅出现两级标题,文史类目录标题为第一章、第一节,理工类目录标题为第一章、1.1。
	
	\item “目录” 居中用黑体二号字,一级标题左对齐用宋体四号字,二级标题与一级标题左空一个字的位置,用宋体小四号字。

\end{enumerate}

%%----------------------
\subsection{正文}

\begin{enumerate}[label=\arabic*)]
	\item 正文是论文的主体,一般由标题、文字叙述、图、表和公式等五个部分构成。写作形式可因科研项目的性质不同而变化,一般可包括理论分析、计算方法、实验装置和测试方法,经过整理加工的实验结果分析和讲座,与理论计算结果的比较以及本研究方法与已有研究方法的比较等。
	
	\item 正文分章节撰写,每章都另起一页。
	
	\item 正文内容使用五号宋体字,行间距为1.5倍行距。
	
\end{enumerate}

%%----------------------
\subsection{标题}
\begin{enumerate}[label=\arabic*)]
	\item 论文标题是以最恰当、最简明的词语反映论文中最重要的特定内容的逻辑组合。标题既要准确地描述内容,又要尽可能地短,一级标题一般不宜超过36个字。标题应该避免使用不常见的缩略词、字符、代号和公式等。
	
	\item 论文标题一般分为三级,文史类与理工类标题格式不同,具体如下:
	
	{\bfseries 文史类}:
	
	第一章(一级标题,居中,黑体三号字,3倍行间距)
	
	第一节(二级标题,居中,黑体四号字,2.5倍行间距)
	
	一、(三级标题,首行缩进2字符,黑体小四号字,2倍行间距)
	如有四五六级标题,可按如下格式:
	
	(一)(四级标题,首行缩进2字符,宋体五号字,2倍行间距)
	1.(五级标题,首行缩进2字符,宋体五号字,2倍行间距)
	
	(1)(六级标题,首行缩进2字符,宋体五号字,2倍行间距)
	
	{\bfseries 理工类}:
	
	第一章(一级标题,居中,黑体三号字,3倍行间距)
	
	1.1(二级标题,左对齐,黑体四号字,2.5倍行间距)
	
	1.1.1(三级标题,左对齐,黑体小四号字,2倍行间距)
	
	\item “参考文献”、“附录”、“致谢”、“个人简介”等标题为居中黑体三号字,3倍行间距;内容使用宋体小四号字,1.5倍行间距。
	
\end{enumerate}

%%----------------------
\subsection{注释}

\begin{enumerate}[label=\arabic*)]
	\item 所有引用、参考、借用的资料数据及他人成果必须标明出处,严禁抄袭、剽窃。
	
	\item 引用文献标注方式应全文统一,文中引用内容使用上标标注,以 \textcircled{1}、\textcircled{2}等为编号标于所引内容最末句右上角,用小五号宋体字;解释内容采用脚注方式,以\textcircled{1}、\textcircled{2}为序号置于页下,用小五号宋体字,两端对齐,单倍行距。
	
	\item 不同页的脚注序号不需要连续编号;同一页几处引用同一文献时,将所有序号一起列出,只标注一次出处。
	
\end{enumerate}

%%----------------------
\subsection{参考文献}

\begin{enumerate}[label=\arabic*)]
	\item 参考文献采用尾注形式,标注于正文结束之后,不得罗列在各章节后。
	
	\item 引用文献标注方式应全文统一,文中引用内容使用上标标注,以 \textcircled{1}、\textcircled{2}等为编号标于所引内容最末句右上角,用小五号宋体字;解释内容采用脚注方式,以\textcircled{1}、\textcircled{2}为序号置于页下,用小五号宋体字,两端对齐,单倍行距。
	
	\item 各类文献资料的排列格式为:
	
	{\bfseries 期刊类}:
	[序号]作者.题目.刊名,出版年份,卷号(期号)
	
	{\bfseries 专(译)著类}:
	[序号]作者.书名(,译者).出版地:出版社,出版年,起止页码
	
	{\bfseries 论文集}:
	[序号]作者.题名,见(英文用In),主编,论文集名,出版地:出版社,出版年,起止页码
	
	{\bfseries 学位论文}:
	[序号]作者,题名,授予单位所在地:授予单位,授予年
	
	{\bfseries 专利}:
	[序号]申请者,专利名,国别,专利文献种类,专利号,出版日期
	
	{\bfseries 技术标准}:
	[序号]发布单位,标准代号,标准顺序号-发布年,标准名称,出版地,出版者,出版日期
	
	{\bfseries 电子文献}:
	[序号]作者.题名.获取或访问路径
	
\end{enumerate}

%%----------------------
\subsection{附录(非必要)}

\begin{enumerate}[label=\arabic*)]
	\item 主要列正文内容过于冗长的公式推导,供查读方便所需的辅助性数学工具或表格;重复性数据图表;论文使用缩写、程序全文及说明等。
	
	\item 附录编号顺序依次为附录1,附录2、附录3……,每个附录应有标题。
	
\end{enumerate}

%%----------------------
\subsection{致谢}

\begin{enumerate}[label=\arabic*)]
	\item 致谢对象仅限对完成课题研究和论文写作过程给予指导和帮助的导师、任课教师、校内外专家、实验技术人员、同学等。
	
	\item 致谢内容以精练的叙述性文字内容为主,用词应含蓄、笼统、简朴,不宜出现感情色彩浓厚和流于俗套的溢美之词,不宜出现图表等。
	
\end{enumerate}

%%----------------------
\subsection{个人简介}

\begin{enumerate}[label=\arabic*)]
	\item 简要介绍自己,内容包括姓名,性别,民族,籍贯,第一学历毕业院校及专业,取得的学位。
	
	\item 在研期间发表的论文,内容包括发表刊物名称,年月、卷册号,页码、论文作者排序及署名单位名称等,罗列论文以发表的时间先后排列。
	
\end{enumerate}

%%============================

\section{编排顺序及打印及装订等要求}

\begin{enumerate}[label=\arabic*)]
	\item 学位论文的编排顺序为外封面、内封面、独创性声明和授权说明、中文摘要、英文摘要、目录、引言/绪论、正文、结论/结语、注释和参考文献、附录、致谢、个人简介等部分。
	
	\item 学位论文内容一律用计算机编辑,用A4规格纸打印,按以上要求装订成册(不得用活页夹装订)。
\end{enumerate}

%%============================

% 说明
% 本LaTeX模板的一般使用说明
\chapter{说明\footnote{章标题中脚注命令测试}}
\label{sec:error2}

%-----------------------------
\section{宏包\footnote{节标题中脚注命令测试}使用}

请将表\ref{tab:tabu_file}中文件清单放在同一目录中,使用\LaTeX{}可以选择TexLive+TeXstudio的方式,安装教程可参看百度经验\footnote{\href{https://jingyan.baidu.com/article/b2c186c83c9b40c46ff6ff4f.html}{https://jingyan.baidu.com/article/b2c186c83c9b40c46ff6ff4f.html}}。

\begin{longtable}{|c|>{\raggedright\arraybackslash}p{8cm}|}
	\caption{北方民族大学学位论文\LaTeX{}模板清单表}\label{tab:tabu_file}
	\endfirsthead
	\caption{北方民族大学学位论文\LaTeX{}模板清单表(续)}
	\endhead
	\hline 
	\rule[0ex]{0pt}{2.5ex} \verb|NMUThesis.tex| & $\triangleright$\LaTeX{}模板(main) \\ 
	\hline 
	\rule[0ex]{0pt}{2.5ex} \verb|NMUThesis.pdf| & $\triangleright$PDF模板样例\\
	\hline 
	\rule[0ex]{0pt}{2.5ex} \verb|nmu.cls |    & $\triangleright$ \LaTeX{}宏模板文件 \\
	\hline 
	\rule[0ex]{0pt}{2.5ex} \verb|GBT7714-2005.bst| & $\triangleright$ 国标参考文献BibTeX样式文件2005 \\
	\hline 
	\rule[0ex]{0pt}{2.5ex} \verb|GBT7714-2015.bst|  & $\triangleright$ 国标参考文献BibTeX样式文件2015 \\
	\hline 
	\rule[0ex]{0pt}{2.5ex} \verb|nmu_logo.png|   & $\triangleright$论文封面北方民族大学校标 \\
	\hline 
	\rule[0ex]{0pt}{2.5ex} \verb|tex/*.tex| & $\triangleright$\LaTeX{}模板样例中的独立章节\\
	\hline 
	\rule[0ex]{0pt}{2.5ex} \verb|figures/*| & $\triangleright$\LaTeX{}模板样例中的插图存放目录\\
	\hline 
	\rule[0ex]{0pt}{2.5ex} \verb|ref.bib |    & $\triangleright$\LaTeX{}模板中的参考文献Bib文件\\
	\hline 
	\rule[0ex]{0pt}{2.5ex} \verb|make.bat|    &$\triangleright$生成NMUThesis.pdf\\
	\hline 
	\rule[0ex]{0pt}{2.5ex} \verb|clean.bat|  & $\triangleright$清理冗余文件\\
	\hline 
\end{longtable}

通过 \verb|\documentclass[<thesis>,<printtype>,<version>]{nmu}|载入宏包:
\begin{itemize}[leftmargin=3cm]
  \item[{\tt thesis} $\triangleright$]  论文类型(thesis),可选值:\\
    a) 学术硕士论文(\verb|master|)[缺省值]\hfill
    b) 专业硕士论文(\verb|professional|)\\
    c) 博士论文(\verb|doctor|)
  \item[{\tt version} $\triangleright$] 论文版本(version),可选值: \\
    a) 盲审版(\verb|blind|)[缺省值]\hfill
    b) 最终版(\verb|ultimate|)
  \item[{\tt printtype} $\triangleright$] 打印属性(printtype),可选值: \\
    a) 单面打印(\verb|onside|)[缺省值]\hfill
    b) 双面打印(\verb|twoside|)
\end{itemize}

模板已内嵌\LaTeX{}工具包:
 {\tt ifthen},{\tt etoolbox},{\tt titletoc},{\tt remreset},{\tt remreset},
 {\tt geometry},{\tt fancyhdr},{\tt setspace},{\tt caption},{\tt float},
 {\tt graphicx},{\tt subfigure},{\tt epstopdf},
 {\tt book\-tabs},{\tt longtable},{\tt multirow},{\tt array}, {\tt enumitem},
 {\tt algorithm2e},{\tt amsmath},{\tt amsthm}, {\tt listings},
 {\tt pifont},{\tt color},{\tt soul}。

模板已内嵌宏:\verb|\highlight{text}|(黄色高亮)。
 
请统一使用UTF-8编码。



%-----------------------------
\section{章节撰写}
本模板支持一下标题级别标题级别

\begin{tabular}{ll}
  \verb|\chapter{章}|              & $\triangleright$ 第一章 \\
  \verb|\chapter*{无章号章}|       & $\triangleright$ 无章号章 \\
  \verb|\chaptera{无章有目录章}|   & $\triangleright$ 无章有目录章 \\
  \verb|\summary|                  & $\triangleright$ 总结\\
  \verb|\appendix|                 & $\triangleright$ 附录\\
  \verb|\acknowledgments|          & $\triangleright$ 致谢\\
  \verb|\biography|                & $\triangleright$ 个人简介\\
  \verb|\section{节}|              & $\triangleright$ 1.1 节\\
  \verb|\subsection{条}|           & $\triangleright$ 1.1.1 条\\
  \verb|\subsubsection{A}|         & $\triangleright$ 1.1.1.1 A\\
  \verb|\paragraph{a}|             & $\triangleright$ 1.1.1.1.1 a\\
  \verb|\subparagraph{a)}|         & $\triangleright$ 1.1.1.1.1.1 a)\\
\end{tabular}

%-----------------------------
\section{选项设置}

\begin{itemize}[leftmargin=3cm]
	\item[{\tt  $\backslash$refcolor} $\triangleright$]  开启/关闭引用编号颜色,包括参考文献,公式,图,表,算法等\\
	\texttt{on}:开启 [默认]\\
	\texttt{off}:关闭
	\item[{\tt $\backslash$beginright} $\triangleright$]  摘要和正文从右侧开始\\
	\texttt{on}:开启 [默认]\\
	\texttt{off}:关闭
	\item[{\tt $\backslash$emptypageword} $\triangleright$]  空白页留字
	\item[{\tt $\backslash$Listfigtab} $\triangleright$]  是否使用图标清单目录\\
	\texttt{on}:开启 [默认]\\
	\texttt{off}:关闭
\end{itemize}


%-----------------------------
\section{注意事项}
\begin{itemize}
  \item[$\triangleright$] 暂无中文斜体;
  \item[$\triangleright$] 中文粗体将转换为楷体;
  \item[$\triangleright$] 若某文献标题中含有特定含义大写字母(“ISODATA”等,详见\cite{Li2017An}),应特别用第二重{}将其括起来才可使其正常表示;
  \item[$\triangleright$] 切换论文版本后,重新生成目录需两次编译;
  \item[$\triangleright$] 行末针对标点的断行不好,例如\ref{sec:error1}处的有个“、”被断在了句首;
  \item[$\triangleright$] \verb|\label{<text>}|中不能使用中文;
  \item[$\triangleright$] 浮动体与正文之间的距离是弹性的;
  \item[$\triangleright$] 命令符与汉字之间请注意加空格以避免undefined错误(pdfLaTeX下好像一般不存在这个问题,主要在XeLaTeX编译环境下发生);
\end{itemize}

%-----------------------------
\section{ToDo}
\begin{itemize}
  \item[$\triangleright$] 数学环境的行间隔;
  \item[$\triangleright$] 模板选项参数选择;
  \item[$\triangleright$] 导入BibTeX参考文献库可通过百度学术或Zotero等(例:如图\ref{fig:3-1}、\ref{fig:3-2});
  \item[$\triangleright$] 表格、图片可使用TeXstudio向导插入(例:如图\ref{subfig:3a}、\ref{subfig:3b});
\end{itemize}

\begin{figure}[tbh!]
	\centering
	\includegraphics[width=0.6\linewidth]{figures/sample/3-1}
	\caption{导入BibTeX参考文献库步骤一}
	\label{fig:3-1}
\end{figure}

\begin{figure}[tbh!]
	\centering
	\includegraphics[width=0.6\linewidth]{figures/sample/3-2}
	\caption{导入BibTeX参考文献库步骤二}
	\label{fig:3-2}
\end{figure}

\begin{figure}[htb!]
	\centering
	\begin{subfigure}[b]{.4\textwidth}
		\centering
		\includegraphics[width=0.7\linewidth]{3-3.png}
		\caption{TeXstudio向导}\label{subfig:3a}
	\end{subfigure}
	\begin{subfigure}[b]{.4\textwidth}
		\centering
		\includegraphics[width=\linewidth]{3-4.png}
		\caption{插入图片}\label{subfig:3b}
	\end{subfigure}
	\caption{使用TeXstudio向导插入图片}\label{fig:3}
\end{figure}
%-----------------------------
\section{意见及问题反馈}

\indent E-mail:wizen\_zhang@163.com\\
\indent GitHub:\href{https://github.com/WizenZhang/NMUThesis/issues}{https://github.com/WizenZhang/NMUThesis/issues}




% 示例
% 本LaTeX模板的使用示例
\chapter{示例}
%==============================
\section{参考文献引用}
参考文献类型:专著[M],会议论文集[C],报纸文章[N],期刊文章[J],学位论文[D],报告[R],标准[S],专利[P],论文集中的析出文献[A]。测试一下上标引用\upcite{Ruel_2012},引用\cite{knuth84,knuth86a,lamport94},还有其它引用\upcite{Ruel_2012,knuth86a,lamport94}.
%--------------------------------
\subsection{数字标注}
\noindent
\begin{tabular}{l@{\quad$\Rightarrow$\quad}l}
  \verb|\cite{knuth86a}| & \cite{knuth86a}\\
  \verb|\citet{knuth86a}| & \citet{knuth86a}\\
  \verb|\citet[chap.~2]{knuth86a}| & \citet[chap.~2]{knuth86a}\\[0.5ex]
  \verb|\citep{knuth86a}| & \citep{knuth86a}\\
  \verb|\citep[chap.~2]{knuth86a}| & \citep[chap.~2]{knuth86a}\\
  \verb|\citep[see][]{knuth86a}| & \citep[see][]{knuth86a}\\
  \verb|\citep[see][chap.~2]{knuth86a}| & \citep[see][chap.~2]{knuth86a}\\[0.5ex]
  \verb|\citet*{knuth86a}| & \citet*{knuth86a}\\
  \verb|\citep*{knuth86a}| & \citep*{knuth86a}\\
\end{tabular}
\par\noindent
\begin{tabular}{l@{\quad$\Rightarrow$\quad}l}
  \verb|\citet{knuth86a,tlc2}| & \citet{knuth86a,tlc2}\\
  \verb|\citep{knuth86a,tlc2}| & \citep{knuth86a,tlc2}\\
  \verb|\cite{knuth86a,knuth84}| & \cite{knuth86a,knuth84}\\
  \verb|\upcite{knuth86a,knuth84}| & \upcite{knuth86a,knuth84}\\
  \verb|\citet{knuth86a,knuth84}| & \citet{knuth86a,knuth84}\\
  \verb|\citep{knuth86a,knuth84}| & \citep{knuth86a,knuth84}\\
  \verb|\cite{knuth86a,knuth84,tlc2}| & \cite{knuth86a,knuth84,tlc2}\\
\end{tabular}

%--------------------------------
\subsection{数字标注-上标形式}
\noindent
\begin{tabular}{l@{\quad$\Rightarrow$\quad}l}
  \verb|\upcite{knuth86a}| & \upcite{knuth86a}\\
  \verb|\upcite{knuth86a,knuth84,tlc2}| & \upcite{knuth86a,knuth84,tlc2}\\
\end{tabular}
\par\noindent
实现源码:\verb|\newcommand{\upcite}[1]{\textsuperscript{\cite{#1}}}|。


%--------------------------------
\subsection{著者-出版年制标}
\citestyle{authoryear}
\noindent
\begin{tabular}{l@{\quad$\Rightarrow$\quad}l}
  \verb|\cite{knuth86a}| & \cite{knuth86a}\\
  \verb|\citet{knuth86a}| & \citet{knuth86a}\\
  \verb|\citet[chap.~2]{knuth86a}| & \citet[chap.~2]{knuth86a}\\[0.5ex]
  \verb|\citep{knuth86a}| & \citep{knuth86a}\\
  \verb|\citep[chap.~2]{knuth86a}| & \citep[chap.~2]{knuth86a}\\
  \verb|\citep[see][]{knuth86a}| & \citep[see][]{knuth86a}\\
  \verb|\citep[see][chap.~2]{knuth86a}| & \citep[see][chap.~2]{knuth86a}\\[0.5ex]
  \verb|\citet*{knuth86a}| & \citet*{knuth86a}\\
  \verb|\citep*{knuth86a}| & \citep*{knuth86a}\\
\end{tabular}
\par\noindent
\begin{tabular}{l@{\quad$\Rightarrow$\quad}l}
  \verb|\citet{knuth86a,tlc2}| & \citet{knuth86a,tlc2}\\
  \verb|\citep{knuth86a,tlc2}| & \citep{knuth86a,tlc2}\\
  \verb|\cite{knuth86a,knuth84}| & \cite{knuth86a,knuth84}\\
  \verb|\citet{knuth86a,knuth84}| & \citet{knuth86a,knuth84}\\
  \verb|\citep{knuth86a,knuth84}| & \citep{knuth86a,knuth84}\\
\end{tabular}
\citestyle{numbers}

%--------------------------------
\subsection{其他形式的标注}
\noindent
\begin{tabular}{l@{\quad$\Rightarrow$\quad}l}
  \verb|\citealt{tlc2}| & \citealt{tlc2}\\
  \verb|\citealt*{tlc2}| & \citealt*{tlc2}\\
  \verb|\citealp{tlc2}| & \citealp{tlc2}\\
  \verb|\citealp*{tlc2}| & \citealp*{tlc2}\\
  \verb|\citealp{tlc2,knuth86a}| & \citealp{tlc2,knuth86a}\\
  \verb|\citealp[pg.~32]{tlc2}| & \citealp[pg.~32]{tlc2}\\
  \verb|\citenum{tlc2}| & \citenum{tlc2}\\
  \verb|\citetext{priv.\ comm.}| & \citetext{priv.\ comm.}\\
\end{tabular}

\noindent
\begin{tabular}{l@{\quad$\Rightarrow$\quad}l}
  \verb|\citeauthor{tlc2}| & \citeauthor{tlc2}\\
  \verb|\citeauthor*{tlc2}| & \citeauthor*{tlc2}\\
  \verb|\citeyear{tlc2}| & \citeyear{tlc2}\\
  \verb|\citeyearpar{tlc2}| & \citeyearpar{tlc2}\\
\end{tabular}

\section{浮动体\footnote{样例参考《浙江大学研究生硕士(博士)学位论文\LaTeX{}模板》}}
在实际撰写文稿的过程中,我们可能会碰到一些占据篇幅较大,但同时又不方便分页的内容。(比如图片和表格,通常属于这样的类型)此时,我们通常会希望将它们放在别的地方,避免页面空间不够而强行置入这些内容导致 overfull vbox 或者大片的空白。此外,因为被放在别的地方,所以,我们通常需要对这些内容做一个简单的描述,确保读者在看到这些大块的内容时,不至于无从下手去理解。同时,因为此类内容被放在别的地方,所以在文中引述它们时,我们无法用「下图」、「上表」之类的相对位置来引述他们。于是,我们需要对它们进行编号,方便在文中引用。

在 \LaTeX{} 中,默认有 figure 和 table 两种浮动体。(当然,你还可以自定义其他类型的浮动体)在这些环境中,可以用 $\backslash$caption\{\} 命令生成上述简短的描述。至于编号,也是用 $\backslash$caption\{\} 生成的。这类编号遵循了 TeX 对于编号处理的传统:它们会自动编号,不需要用户操心具体的编号数值。 至于「别的地方」是哪里,\LaTeX{} 为浮动体启用了所谓「位置描述符」的标记。基本来说,包含以下几种:

h - 表示 here。此类浮动体称为文中的浮动体(in-text floats)。

t - 表示 top。此类浮动体会尝试放在一页的顶部。

b - 表示 bottom。此类浮动体会尝试放在一页的底部。

p - 表示 float page,浮动页。此类浮动体会尝试单独成页。

\LaTeX{} 会将浮动体与文本流分离,而后按照位置描述符,根据相应的算法插入 \LaTeX{} 认为合适的位置。

\subsection{插图测试}
如图\ref{fig:first_image_tset}是对此模版的第一张插图测试。

\begin{figure}[htbp]
	\centering
	\includegraphics[width = 0.5\linewidth]{Chapter1.png}
	\caption{第一张插图测试}\label{fig:first_image_tset}
\end{figure}

以下是一段对这些插图来历的介绍,引用自知乎专栏All about TeXnique中夏晓昊的文章\href{http://zhuanlan.zhihu.com/LaTeX/19669122}{《The TeXbook导读:从那头(多图杀猫的)狮子说起》}。

在The TeXbook中,有着一系列的以狮子为主题的插图。这些插图的作者是Duane Bibby。也是从The TeXbook开始,不少TeX书也采取了以狮子为主的插图,作者也是Duane Bibby。另外,每年的TUG(TeX Users Group)年会都会有一张以狮子为主题的logo,这只狮子已经是社区的吉祥物了。

为什么选择狮子呢?Yannis Haralambous写道(原文法语,此为转译后的英文):Not for nothing is TeX represented by a lion. Donald Knuth has told us that lions are to him the guardians of libraries in the United States because there is a statue of a lion in front of the entrance of each large library there. Guardian of libraries, guardian of the Book—is that not indeed what TeX ultimately aspires to be? 或许吧。 (顺便说一句,TeX和MetaFont都用了狮子,TeX是公狮子,MetaFont是母狮子,多么和谐的一对啊。如果你还是忽略MetaFont的存在,那你还没有认识到它的重要性。)

作为插图,首要的一点就是贴切,然后是有趣。在TeX社区里面,have fun是一个很重要的词组,也有人说Happy TeXing。我知道有不少人不喜欢TeX,但是能有什么理由呢?如果你用不到它,那么浅尝辄止即可。如果你会用到很频繁,最好慢慢修炼做到精通。如果你只是偶尔用到,那么可以搬个模版什么的,甚至也可以找人帮你(不要指望别人会用足够的空闲时间来帮你,他没有这个义务,请支付报酬,最少也得请吃个饭吧)。下面的插图,是TeX TeXbook中的,我也希望这个假期,能有人有空来看看这本书。即使不能把所有的东西都看懂,那么也会对TeX的设计有了一定的了解,拿到扳手就好。

\subsection{表格测试}
在这里推荐制表采用功能强大的tabu宏包以取代其它制表宏包。具体tabu宏包的使用说明参见tabu宏包的说明文档。

以下节分别用来测试各种表格环境如,tabular,tabu,longtabu等,还有对caption格式的修改和测试。以下表格样式全部采用三线表。

\subsection{array宏包tabular表格环境测试}
如表\ref{tab:first_table_test}是对array宏包的tabular表格环境测试。
\begin{table}[htbp]
	\centering
	\caption{这是一个用tabular环境的测试用的表格}\label{tab:first_table_test}
	\begin{tabular}{lrr}
		\toprule
		\textbf{行星}     & \textbf{赤道半径}km & \textbf{公转周期}d \\
		\midrule
		水星     & 2.439  & 87.9 \\
		金星     & 6.1    & 224.682 \\
		地球     & 6378.14 & 365.24 \\
		\bottomrule
	\end{tabular}%
\end{table}

\subsection{tabu宏包表格环境测试}
如表\ref{tab:tabu_test_1}是对tabu宏包的tabu表格环境测试。在这里表格命令与表\ref{tab:first_table_test}的命令相同,只是tabular环境改成了tabu环境。
\begin{table}[htbp]
	\centering
	\caption{这是一个用tabu环境的测试用的表格}\label{tab:tabu_test_1}
	\begin{tabu}{lrr}
		\toprule
		\textbf{行星}     & \textbf{赤道半径}km & \textbf{公转周期}d \\
		\midrule
		水星     & 2.439  & 87.9 \\
		金星     & 6.1    & 224.682 \\
		地球     & 6378.14 & 365.24 \\
		\bottomrule
	\end{tabu}%
\end{table}

表\ref{tab:tabu_test_2}对tabu to表格的x列模式进行测试。在表格导言区中设置为X[1]X[2]X[2],表示这三列表格的列宽比值为1:2:2,总的表格宽度由tabu to环境设置,这里设置为0.6\textbackslash linewidth。相比于tabular环境,tabu环境的列宽设置方便许多。
\begin{table}[htbp]
	\centering
	\caption{tabu环境测试表格---X列模式}\label{tab:tabu_test_2}
	\begin{tabu} to 0.6\linewidth{X[1]X[2]X[2]}
		\toprule
		\textbf{行星}     & \textbf{赤道半径}km & \textbf{公转周期}d \\
		\midrule
		水星     & 2.439  & 87.9 \\
		金星     & 6.1    & 224.682 \\
		地球     & 6378.14 & 365.24 \\
		\bottomrule
	\end{tabu}%
\end{table}

如表\ref{tab:tabu_test_3}是longtabu环境测试表格。longtabu环境不能用在table浮动体环境中。根据GB/T 7713.1-2006规定:如果某个表需要转页接排,在随后的各页上应重复表的编号。编号后跟标题(可省略)和“(续)”,置于表上方。续表应重复表头。

特别需要注意的是,longtabu是基于longtable宏包开发的,所以在nmu.cls文件中已经插入了longtable宏包。longtable环境的所有功能都可以在longtabu中使用,如\textbackslash endhead,\textbackslash endfirsthead,\textbackslash endfoot,\textbackslash endlastfoot,和\textbackslash caption等。具体用法请参见longtable和tabu宏包的相应文档。
\begin{longtabu}{cccccc}
	\caption{2018年6月全球编程语言TIOBE排行榜}\label{tab:tabu_test_3}\\
	\toprule
	Jun 2018   & Jun 2017 & Change & Programming Language & Ratings &Change\\
	\midrule%
	\endfirsthead
	\caption{2018年6月全球编程语言TIOBE排行榜(续)}\\
	\toprule
	Jun 2018   & Jun 2017 & Change & Programming Language & Ratings &Change \\
	\midrule%
	\endhead
	\bottomrule%
	\endfoot
1	&	1	&		&	Java	&	15.368$\%$	&	+0.88$\%$	\\
2	&	2	&		&	C	&	14.936$\%$	&	+8.09$\%$	\\
3	&	3	&		&	C++	&	8.337$\%$	&	+2.61$\%$	\\
4	&	4	&		&	Python	&	5.761$\%$	&	+1.43$\%$	\\
5	&	5	&		&	C$\#$	&	4.314$\%$	&	+0.78$\%$	\\
6	&	6	&		&	Visual Basic .NET	&	3.762$\%$	&	+0.65$\%$	\\
7	&	8	&	$\uparrow$	&	PHP	&	2.881$\%$	&	+0.11$\%$	\\
8	&	7	&	$\downarrow$	&	JavaScript	&	2.495$\%$	&	-0.53$\%$	\\
9	&	-	&	$\uparrow$	&	SQL	&	2.339$\%$	&	+2,34$\%$	\\
10	&	14	&	$\uparrow$	&	R	&	1.452$\%$	&	-0.70$\%$	\\
11	&	11	&		&	Ruby	&	1.253$\%$	&	-0.97$\%$	\\
12	&	18	&	$\uparrow$	&	Objective-C	&	1.181$\%$	&	-0.78$\%$	\\
13	&	16	&	$\uparrow$	&	Visual Basic	&	1.154$\%$	&	-0.86$\%$	\\
14	&	9	&	$\downarrow$	&	Perl	&	1.147$\%$	&	-1.16$\%$	\\
15	&	12	&	$\downarrow$	&	Swift	&	1.145$\%$	&	-1.06$\%$	\\
16	&	10	&	$\downarrow$	&	Assembly language	&	0.915$\%$	&	-1.34$\%$	\\
17	&	17	&		&	MATLAB	&	0.894$\%$	&	-1.10$\%$	\\
18	&	15	&	$\downarrow$	&	Go	&	0.879$\%$	&	-1.17$\%$	\\
19	&	13	&	$\downarrow$	&	Delphi/Object Pascal	&	0.875$\%$	&	-1.28$\%$	\\
20	&	20	&		&	PL/SQL	&	0.848$\%$	&	-0.72$\%$	\\
21	&		&		&	SAS	&	1.102$\%$	&		\\
22	&		&		&	Dart	&	0.799$\%$	&		\\
23	&		&		&	COBOL	&	0.685$\%$	&		\\
24	&		&		&	D	&	0.545$\%$	&		\\
25	&		&		&	Lua	&	0.519$\%$	&		\\
26	&		&		&	ABAP	&	0.463$\%$	&		\\
27	&		&		&	Fortran	&	0.459$\%$	&		\\
28	&		&		&	Transact-SQL	&	0.427$\%$	&		\\
29	&		&		&	Scratch	&	0.398$\%$	&		\\
30	&		&		&	Scala	&	0.377$\%$	&		\\
31	&		&		&	Apex	&	0.362$\%$	&		\\
32	&		&		&	Prolog	&	0.349$\%$	&		\\
33	&		&		&	Ada	&	0.340$\%$	&		\\
34	&		&		&	Lisp	&	0.338$\%$	&		\\
35	&		&		&	F$\#$	&	0.338$\%$	&		\\
36	&		&		&	LabVIEW	&	0.331$\%$	&		\\
37	&		&		&	Julia	&	0.301$\%$	&		\\
38	&		&		&	Kotlin	&	0.278$\%$	&		\\	
\end{longtabu}%

\subsection{子图}
这里子图的排版推荐使用subcaption宏包,不再推荐使用subfig宏包,更不推荐使用subfigure宏包。值得注意的是,在nmu.cls文件中已经写入了subcaption宏包,而且subcaption宏包与subfigure和subfig宏包是相互冲突的。因此,如果你还想使用subfig宏包而不想使用subcaption宏包,请自己到nmu.cls文件的相关位置更改,具体的使用及修改方法参见相应的宏包说明文档。不过在这里还是不推荐直接去更改nmu.cls文档,除非你对\LaTeX{} 的相关命令很清楚,知道自己在改什么,并且不会对其他格式产生影响。

具体的subcaption宏包使用方法我这里不仔细介绍,以下只是对subcaption进行一些简单的测试,主要是格式调整和交叉引用。

如图\ref{fig:subfig_test1}是有两张子图的模式,对子图进行交叉引用,如图\ref{subfig:1a}和图\ref{subfig:1b}。

\begin{figure}[htbp]
	\centering
	\begin{subfigure}[b]{.4\textwidth}
		\centering
		\includegraphics[width = \textwidth]{Chapter2.png}
		\caption{书籍排版与普通排版}\label{subfig:1a}
	\end{subfigure}
	\quad
	\begin{subfigure}[b]{.4\textwidth}
		\centering
		\includegraphics[width = \textwidth]{Chapter3.png}
		\caption{\TeX 的控制系列}\label{subfig:1b}
	\end{subfigure}
	\caption{子图模式测试1:2张图}\label{fig:subfig_test1}
\end{figure}

如图\ref{fig:subfig_test2}是有四张子图的模式,对子图进行交叉引用,如图\ref{subfig:2a}、图\ref{subfig:2b}、图\ref{subfig:2c}和图\ref{subfig:2d}。

\begin{figure}[htbp]
	\centering
	\begin{subfigure}[b]{.4\textwidth}
		\centering
		\includegraphics[width = \textwidth]{Chapter4.png}
		\caption{字体}\label{subfig:2a}
	\end{subfigure}
	\begin{subfigure}[b]{.4\textwidth}
		\centering
		\includegraphics[width = \textwidth]{Chapter5.png}
		\caption{编组}\label{subfig:2b}
	\end{subfigure}
	\begin{subfigure}[b]{.4\textwidth}
		\centering
		\includegraphics[width = \textwidth]{Chapter6.png}
		\caption{运行\TeX}\label{subfig:2c}
	\end{subfigure}
	\begin{subfigure}[b]{.4\textwidth}
		\centering
		\includegraphics[width = \textwidth]{Chapter7.png}
		\caption{\TeX 工作原理}\label{subfig:2d}
	\end{subfigure}
	\caption{子图模式测试2:4张图}\label{fig:subfig_test2}
\end{figure}

\section{算法环境}

模板中使用 \texttt{algorithm2e} 宏包实现算法环境。关于该宏包的具体用法请阅读宏包的官方文档。\\
\centerline{-----------$\downarrow$-----------Space Check-----------$\downarrow$-----------}
\begin{algorithm}[!h]
  %\SetAlgoLined
  %\SetAlgoVlined
  \caption{A How to (plain).}
  \KwData{this text}
  \KwResult{how to write algorithm with \LaTeX2e{} }
  
  initialization\;
  \While{not at end of this document}{
    read current\;
    \eIf{understand}{
      go to next section\;
      current section becomes this one\;
    }{
      go back to the beginning of current section\;
    }
  }
\end{algorithm}

\centerline{-----------$\uparrow$-----------Space Check-----------$\uparrow$-----------}

\centerline{-----------$\downarrow$-----------Space Check-----------$\downarrow$-----------}
\RestyleAlgo{ruled}
\begin{algorithm}[!h]
  \caption{A How to (ruled).}
  \KwData{this text}
  \KwResult{how to write algorithm with \LaTeX2e{} }
  
  initialization\;
  \While{not at end of this document}{
    read current\;
    \eIf{understand}{
      go to next section\;
      current section becomes this one\;
    }{
      go back to the beginning of current section\;
    }
  }
\end{algorithm}

\centerline{-----------$\uparrow$-----------Space Check-----------$\uparrow$-----------}

\RestyleAlgo{boxed}
\begin{algorithm}[!h]
  \caption{A How to (boxed).}
  \KwData{this text}
  \KwResult{how to write algorithm with \LaTeX2e{} }
  
  initialization\;
  \While{not at end of this document}{
    read current\;
    \eIf{understand}{
      go to next section\;
      current section becomes this one\;
    }{
      go back to the beginning of current section\;
    }
  }
\end{algorithm}

\RestyleAlgo{boxruled}
\begin{algorithm}[!h]
  \caption{A How to (boxruled).}
  \KwData{this text}
  \KwResult{how to write algorithm with \LaTeX2e{} }
  
  initialization\;
  \While{not at end of this document}{
    read current\;
    \eIf{understand}{
      go to next section\;
      current section becomes this one\;
    }{
      go back to the beginning of current section\;
    }
  }
\end{algorithm}

\section{数学环境}

\subsection{数学符号}

模板定义了一些正体(upright)的数学符号:
\begin{center}
  \begin{tabular}{rl}
    \toprule
    符号                 & 命令 \\
    \midrule
    常数$\eu$     & \verb|\eu| \\
    复数单位$\iu$ & \verb|\iu| \\
    微分符号$\diff$ & \verb|\diff| \\
    $\argmax$         & \verb|\argmax| \\
    $\argmin$         & \verb|\argmin| \\
    \bottomrule
  \end{tabular}
\end{center}

更多的例子:
\begin{equation}
\eu^{\iu\pi} + 1 = 0
\end{equation}
\begin{equation}
\frac{\diff^2u}{\diff t^2} = \int f(x) \diff x
\end{equation}
\begin{equation}
\argmin_x f(x)
\end{equation}

\subsection{定理、引理和证明}

\begin{definition}
  If the integral of function $f$ is measurable and non-negative, we define
  its (extended) \textbf{Lebesgue integral} by
  \begin{equation}
  \int f = \sup_g \int g,
  \end{equation}
  where the supremum is taken over all measurable functions $g$ such that
  $0 \leq g \leq f$, and where $g$ is bounded and supported on a set of
  finite measure.
\end{definition}

\begin{example}
  Simple examples of functions on $\mathbf{R}^d$ that are integrable
  (or non-integrable) are given by
  \begin{equation}
  f_a(x) =
  \begin{cases}
  |x|^{-a} & \text{if } |x| \leq 1,\\
  0 & \text{if } x > 1.
  \end{cases}
  \end{equation}
  \begin{equation}
  F_a(x) = \frac{1}{1 + |x|^a}, \qquad \text{all } x \in \mathbf{R}^d.
  \end{equation}
  Then $f_a$ is integrable exactly when $a < d$, while $F_a$ is integrable
  exactly when $a > d$.
\end{example}

\begin{lemma}[Fatou]
  Suppose $\{f_n\}$ is a sequence of measurable functions with $f_n \geq 0$.
  If $\lim_{n \to \infty} f_n(x) = f(x)$ for a.e. $x$, then
  \begin{equation}
  \int f \leq \liminf_{n \to \infty} \int f_n.
  \end{equation}
\end{lemma}

\begin{remark}
  We do not exclude the cases $\int f = \infty$,
  or $\liminf_{n \to \infty} f_n = \infty$.
\end{remark}

\begin{corollary}
  Suppose $f$ is a non-negative measurable function, and $\{f_n\}$ a sequence
  of non-negative measurable functions with
  $f_n(x) \leq f(x)$ and $f_n(x) \to f(x)$ for almost every $x$. Then
  \begin{equation}
  \lim_{n \to \infty} \int f_n = \int f.
  \end{equation}
\end{corollary}

\begin{proposition}
  Suppose $f$ is integrable on $\mathbf{R}^d$. Then for every $\epsilon > 0$:
  \begin{enumerate}
    \renewcommand{\theenumi}{\roman{enumi}}
    \item There exists a set of finite measure $B$ (a ball, for example) such that
    \begin{equation}
    \int_{B^c} |f| < \epsilon.
    \end{equation}
    \item There is a $\delta > 0$ such that
    \begin{equation}
    \int_E |f| < \epsilon \qquad \text{whenever } m(E) < \delta.
    \end{equation}
  \end{enumerate}
\end{proposition}

\begin{theorem}
  Suppose $\{f_n\}$ is a sequence of measurable functions such that
  $f_n(x) \to f(x)$ a.e. $x$, as $n$ tends to infinity.
  If $|f_n(x)| \leq g(x)$, where $g$ is integrable, then
  \begin{equation}
  \int |f_n - f| \to 0 \qquad \text{as } n \to \infty,
  \end{equation}
  and consequently
  \begin{equation}
  \int f_n \to \int f \qquad \text{as } n \to \infty.
  \end{equation}
\end{theorem}

\begin{proof}
  Trivial.
\end{proof}


\subsection{自定义}

\newtheorem*{axiomofchoice}{Axiom of choice}
\begin{axiomofchoice}
  Suppose $E$ is a set and ${E_\alpha}$ is a collection of
  non-empty subsets of $E$. Then there is a function $\alpha
  \mapsto x_\alpha$ (a ``choice function'') such that
  \begin{equation}
  x_\alpha \in E_\alpha,\qquad \text{for all }\alpha.
  \end{equation}
\end{axiomofchoice}

\newtheorem{observation}{Observation}[chapter]
\begin{observation}
  Suppose a partially ordered set $P$ has the property
  that every chain has an upper bound in $P$. Then the
  set $P$ contains at least one maximal element.
\end{observation}
\begin{proof}[A concise proof]
  Obvious.
\end{proof}

\newtheorem{observationvar2}[observation]{Observationvar2}
\begin{observationvar2}
  Suppose a partially ordered set $P$ has the property
  that every chain has an upper bound in $P$. Then the
  set $P$ contains at least one maximal element.
\end{observationvar2}
\begin{proof}[A concise proof]
  Obvious.
\end{proof}

% 总结
% 总结
\summary
%%%%%%%%%%%%%%%%%%%%%%%%%%%%%%%%%%%%%%%%%%%%%%%%%%%%%%%%
% 内容由此开始

学位论文的结论单独作为一章,但不加章号。如果不可能导出应有的结论,也可以没有结论而进行必要的讨论。

\begin{enumerate}[label=\arabic*)]
	\item 结论是对论文主要研究结果、论点的提炼与概括,主要阐述自己的创造性工作及所取得的研究成果在本学科学术领域中的地位、作用、意义、及本文研究的不足之处或未予解决的遗留问题。
	
	\item 结论要准确、完整、明确、精炼,对自己研究的评价要实事求是;要严格区分自己取得的成果与导师及他人的科研成果的界限。
\end{enumerate}

\par * 嗯,这就是你的论文了 * \par

% 内容到结束		
%%%%%%%%%%%%%%%%%%%%%%%%%%%%%%%%%%%%%%%%%%%%%%%%%%%%%%%%
\ifthenelse{\equal{\@beginright}{off}}{\clearpage}{\clearautopage}

% 参考文献
% [参考文献]
\reference
% 2015版国标GBT7714-2015
% 2005版国标GBT7714-2005
\begin{spacing}{1.3}\xiaosi %参考文献内容为小四号
\bibliographystyle{GBT7714-2015}
\bibliography{ref}
\nocite{*}%显示未被引用的参考文献库
\end{spacing}



% 附录
% [附录]
\appendix
\begin{spacing}{1.3}\xiaosi %附录内容为小四号
%%%%%%%%%%%%%%%%%%%%%%%%%%%%%%%%%%%%%%%%%%%%%%%%%%%%%%%%
% 内容由此开始

\begin{enumerate}[label=\arabic*)]
\item 主要列正文内容过于冗长的公式推导,供查读方便所需的辅助性数学工具或表格;重复性数据图表;论文使用缩写、程序全文及说明等。

\item 附录编号顺序依次为附录1、附录2、附录3……,每个附录应有标题。
\end{enumerate}
下列内容可以作为附录:

\begin{enumerate}[label=\arabic*)]
\item 为了整篇论文材料的完整,但编入正文又有损于编排的条理和逻辑性,这一材料包括比正文更为详尽的信息、研究方法和技术更深入的叙述,建议可以阅读的参考文献题录,对了解正文内容有用的补充信息等;
\item 由于篇幅过大或取材于复制品而不便于编入正文的材料;
\item 不便于编入正文的罕见的珍贵或需要特别保密的技术细节和详细方案(这中情况可单列成册);
\item 对一般读者并非必要阅读,但对专业同行有参考价值的资料;
\item 某些重要的原始数据、过长的数学推导、计算程序、框图、结构图、注释、统计表、计算机打印输出文件等。
\end{enumerate}

\par * 嗯,自由发挥吧 * \par

% 内容到此结束
%%%%%%%%%%%%%%%%%%%%%%%%%%%%%%%%%%%%%%%%%%%%%%%%%%%%%%%%
\end{spacing}

% 致谢(盲审版不显示)
% [致谢]
% \acknowledgments = \chapter{致谢}
\acknowledgments
\thispagestyle{fancy}%添加页眉页脚
致谢中主要感谢指导教师和在学术方面对论文的完成有直接贡献及重要帮助的团体和人士,以及感谢给予转载和引用权的资料、图片、文献、研究思想和设想的所有者。致谢中还可以感谢提供研究经费及实验装置的基金会或企业等单位和人士。致谢辞应谦虚诚恳,实事求是,切记浮夸与庸俗之词。

\begin{enumerate}[label=\arabic*)]
	\item 致谢对象仅限对完成课题研究和论文写作过程给予指导和帮助的导师、任课教师、校内外专家、实验技术人员、同学等。
	\item 致谢内容以精练的叙述性文字内容为主,用词应含蓄、笼统、简朴,不宜出现感情色彩浓厚和流于俗套的溢美之词,不宜出现图表等。
\end{enumerate}

\par * 嗯,感谢完所有人之后,也请记得感谢一下自己 * \par

% 个人简介(盲审版不显示)
% [个人简介]
% \biography = \chapter{个人简介}
\biography
\ifthenelse{\equal{\Show{}}{\show{}}}{% 是否显示[个人简介]章
\begin{spacing}{1.3}\xiaosi %致谢内容为小四号
\thispagestyle{fancy}%添加页眉页脚
%%%%%%%%%%%%%%%%%%%%%%%%%%%%%%%%%%%%%%%%%%%%%%%%%%%%%%%%
% 内容由此开始

\begin{enumerate}[label=\arabic*)]
	\item 简要介绍自己,内容包括姓名,性别,民族,籍贯,第一学历毕业院校及专业,取得的学位。
	
	\item 在研期间发表的论文,内容包括发表刊物名称,年月、卷册号,页码、论文作者排序及署名单位名称等,罗列论文以发表的时间先后排列。
\end{enumerate}

\vspace{5cm}

This is \NMUThesis{}, Happy TeXing! --- from Wizen Zhang.

% 内容由此结束		
%%%%%%%%%%%%%%%%%%%%%%%%%%%%%%%%%%%%%%%%%%%%%%%%%%%%%%%%
\end{spacing}		
}{}

\end{document}